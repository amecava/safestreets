\section{Requirements traceability}

    \subsection{Functional requirements}

        In the \autoref{tbl:mappingGoalsOnComponent} is presented a mapping correspondence between the requirements defined in the RASD
        related to each goal and the components identified in the server component diagram.\\

        \begin{longtable}{p{0.7\linewidth}p{0.3\linewidth}}
            \toprule
                \textbf{Requirements of goal} & \textbf{Components}\\
            \midrule
                \textbf{G1} Allow guest users to register to the system 
                    & \mbox{UserAppServer} \mbox{AccessHandler} \mbox{DataProvider}\\
            \midrule
                \textbf{G2} Allow registered users to authenticate to the system 
                    & \mbox{UserAppServer} \mbox{AccessHandler} \mbox{DataProvider}\\
            \midrule 
                \textbf{G3} Allow users to transfer data to the system describing occurred violations,
                 including the suitable metadata to describe the submitted violation 
                 & \mbox{UserAppServer} \mbox{SubmissionHandler} \mbox{DataProvider}\\
            \midrule
                \textbf{G4} Ensure that the chain of custody of the information provided by the users is never broken, 
                and the information is never altered or manipulated 
                & \mbox{UserAppServer} \mbox{SubmissionHandler} \mbox{DataProvider}  \mbox{APIHandler}\\
            \midrule
                \textbf{G5} Allow the system to retrieve data about the accidents that occur on the territory and data 
                about issued tickets via the municipality provided service 
                & \mbox{APIHandler} \mbox{DataProvider}\\
            \midrule
                \textbf{G6} Allow the system to cross the information submitted by the users and the information retrieved 
                from the municipality to build and provide statistics 
                & \mbox{StatisticsHandler} \mbox{DataProvider} \mbox{APIHandlers}\\
            \midrule
                \textbf{G7} Allow users to consult a map highlighting the streets (and the areas) with the highest frequency 
                of violations, the identified potentially unsafe areas and view statistics about previously stored violations 
                & \mbox{UserAppServer} \mbox{MapHandler} \mbox{DataProvider} \mbox{StatisticsHandler}\\
            \midrule
                \textbf{G8} Allow municipality to consult the system data and receive suggestions on possible interventions via 
                a restrict access API 
                & \mbox{APIHandler} \mbox{DataProvider}\\
            \bottomrule
            \caption{\label{tbl:mappingGoalsOnComponent}Mapping goals on components}
        \end{longtable} 

        \todo{goal 9}

    \subsection{Non functional requirements}
    \todo{RASD?}

        \paragraph{Performance requirements}
        To guarantee a short response time we have tried to decouple components in order to enable an instance pooling management of components by the EJB container and so an as much as possible concurrent management of requests. \\
        During all the design process we also have kept in mind the scalability of the software w.r.t. the number of cars trying to keep a linear complexity factor and to reduce car-dependent activities where it is possible.

        \paragraph{Availability}
        To ensure availability requirements server will be running 24 hours for day. The architecture is designed with the purpose of enabling a redundancy architecture if availability constraints would make it necessary.

        \paragraph{Security}
        Security protocols are used for transmission and storage of sensitive data.
        Maintenance API is only accessible through token mechanisms.
        Customer Care Server is accessed over a VPN to ensure security. \todo{fix}

        \paragraph{Portability}
        Users access the system through a web-based application that enables the portability and a cross-operative system compatibility.
 