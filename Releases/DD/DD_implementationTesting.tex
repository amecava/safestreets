\section{Implementation, integration and test plan}

    \subsection{Overview}

        Our system's client-server architecture has two main subsystems:
        \begin{itemize}           
            \item MobileApp
            \item AppServer
        \end{itemize}
        
        The AppServer will be implemented following a bottom-up strategy since it will be built on various modules that cooperate in order
        to ensure that the goals and standards stated in the \emph{RASD}\cite{RASD} will be met.\\
        The AppServer relies on the GIS and the Municipality API, which will be considered reliable and accurate services and will be
        tested only "server side" (i.d. checking that our system can exploit them properly).\\
        For what it concerns the the AppServer, each submodule will be implemented and tested with formal methods before being integrated 
        with the other modules it has to deal with, since integrating a malfunctioning module with other can only lead to bigger problems 
        and the effort and time needed to fix them could increase exponentially.\\
        To better understand the decisions we made about implementation, testing and integration here are two tables that list the features
        we want to offer linked with the relevance we gave them considering the difficulty of the implementation and the importance for 
        the user.\\

        \begin{table}[h!]
            \begin{center}
            
            \begin{tabular}{l|c|r} % <-- Alignments: 1st column left, 2nd middle and 3rd right, with vertical lines in between
                \textbf{Feature} & \textbf{Importance} & \textbf{Difficulty of implementation}\\
                
                \hline
                Sign up and login & Low & Low\\
                Consult personal data & Low & Low\\
                Upload a picture & High & Medium\\
                Consult the map & Medium & High\\
                See statistics & Medium & High\\

            \end{tabular}
            \caption{Users features}
            \label{tab:table1}
            \end{center}
        \end{table}


      
        \begin{table}[h!]
            \begin{center}
            
            \begin{tabular}{l|c|r} % <-- Alignments: 1st column left, 2nd middle and 3rd right, with vertical lines in between
                \textbf{Feature} & \textbf{Importance} & \textbf{Difficulty of implementation}\\
                
                \hline
                Authenticate & Low & Low\\
                Access the API & High & Low\\
                Retrieve violations' data & High & Medium\\
                Retrieve statistics & Medium & Medium\\
                Retrieve tickets's data & High & High\\
                Share data & Medium & Medium\\

            \end{tabular}
            \caption{Municipality features}
            \label{tab:table1}
            \end{center}
        \end{table}

    \subsection{Component integration}

