\setlength{\parindent}{4ex}
\setlength{\parskip}{1ex}

\section{User interface design}
In this section are presented or referenced some mockups of the main features and related user interfaces the system is supposed to offer to the user and to the validation service through the proper web-based application.

Mockups show how the user interface is supposed to offer to the user the possibility to interact and make request to the system (obviously such user's interactions will result in a client-server communication of the user/validation service app view with the related server component based on the protocol chosen for that communication).

The main goal of our mockups design process is to build an interface that clearly distinguish functionalities offered by the system taking into account the architectural decoupling offered by the taken design choices.

\subsection{User app}
The user app mockups have been discussed and presented in the User Interfaces section of the \emph{SafeStreets: Requirements Analysis and Specification Document} \cite{RASD}. For further implementation details refer to the \hyperref[sec:implementationChoices]{Implementation Choices} section of this document.

\subsection{Validation Service app}
The validation service app has a simple interface in order to provide features to validation service operator in a clear and simple way. This interface must be optimized for a desktop monitor size.

\subsubsection{Home page}
All main features are accessible directly from the home page to provide a rapid and intuitive access to them. \\
From the home page the operator can:
\begin{itemize}
	\item Visualize the pending reports of possibly occurred accidents
	\item Select a specific report to visualize all the stored data regarding it
	\item Visualize the result of the image recognition alghoritms associated to the selected report
	\item Furtherly support the image recognition by manually modifying the report license plate if not present, wrongly typed by the user, or totally not recognized
	\item Mark and unmark reports as valid in order to filter valid and not valid user’s submissions
\end{itemize}

\clearpage
 
 \begin{figure}[ht!]
			\centering
			\includegraphics[width=0.8\linewidth]{mockups/customerCare1}
			\caption{
				\label{fig:cc1} 
				\emph{Validation service home page} mockup
			}
		\end{figure}
		
\subsubsection{User's information}

From the home page the operator could access the list of all the not valid reports and valid violations.

\begin{figure}[ht!]
	\centering
	\includegraphics[width=0.8\linewidth]{mockups/customerCare2}
	\caption{
		\label{fig:cc2} 
		\emph{Validation service reports and violations list} mockup
	}
\end{figure}