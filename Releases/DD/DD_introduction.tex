\section{Introduction}
\subsection{Purpose of this document}
In this document we are going to describe software design and architecture of the SafeStreets system.\\
The software architecture of a system is the structure or structures of the system, which comprise software elements, the externally visible properties of those elements, and the relationships among them.\cite{SoftwareArch}

\subsection{Scope}
SafeStreets is a crowd-­sourced application that intends to provide users with the possibility to notify authorities when traffic violations occur, and in particular parking violations. 
	
	The system allows users to send pictures of violations, including suitable metadata, to authorities. Examples of violations are vehicles parked in the middle of bike lanes or in places reserved for people with disabilities, double parking, and so on. In addition, the system allows users to mine the previously stored information, for example by highlighting the streets (and the areas) with the highest frequency of violations, or by showing different types of statistics built upon the collected data.
	
	The system will also provide a communication interface to the municipality's provided service to create a secure bridge for data transfer. This connection will enable SafeStreets to cross its data with municipality's to make analysis and build different types of statistics. Moreover the system will offer back to the municipality the possibility to retrieve information about the violations in order to generate traffic tickets from it and receive suggestions on possible interventions.\cite{RASD}
\subsection{Glossary}
The \emph{SafeStreets: Requirements Analysis and Specification Document}\cite{RASD} should be referenced for terms not defined in this section.
\subsubsection{Definitions}
	
\subsubsection{Acronyms}
	\begin{description}
		\item [RASD:] Requirements Analysis and Specification Document
		\item [DD:] Design Document
		\item [API:] Application Programming Interface
		\item [GPS:] Global Position System
		\item [DB:] DataBase
		\item [DBMS:] DataBase Management System
		\item [GIS:] Geographic Information System
		\item [ER:] Entity Relationship Model
		\item [XML:] eXtensible Markup Language
		\item [REST API:] REpresentational State Transfer API
		\item [JAX-RS:] JAVA API for REST Web Services
		\item [ISP:] Internet Service Provider
		\item [ARP:] Address Resolution Protocol
	\end{description}
\subsubsection{Abbreviations}
	\begin{description}
		\item [m:] meters (with multiples and submultiples)
		\item [w.r.t.:] with respect to
		\item [i.d.:] id est
		\item [i.f.f.:] if and only if
		\item [e.g.:] exempli gratia
		\item [etc.:] et cetera
	\end{description}

\subsection{Reference documents}
Context, domain assumptions, goals, requirements and system interfaces are all described in the \emph{SafeStreets: Requirements Analysis and Specification Document}.\cite{RASD}\\
Others references are:
\begin{itemize}
	\item IEEE Std 1016-2009 Standard for Information Technology, Systems Design, Software Design Descriptions
\end{itemize}

\subsection{Document overview}
This document is structured as
\begin{enumerate}
	\item \textbf{Introduction}: it provides an overview of the entire document
	\item \textbf{Architectural design}: it describes different views of components and their interactions
	\item \textbf{User interface design}: it provides an overview on how the user interfaces of our system will look like
	\item \textbf{Requirements traceability}: it explains how the requirements we have defined in the RASD map to the design elements that we have defined in this document.
	\item \textbf{Implementation, integration and test plan}: it focuses on the implementation and testing strategies that will be adopted to build the system
	\item \textbf{Appendices}: it contains references, software and tools used and hours of work per each team member
\end{enumerate}